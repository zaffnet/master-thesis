\chapter{Deployment Artifacts}
\label{app:deployment-artifacts}

\begin{lstlisting}[language=bash, caption={Dockerfile}, label={lst:dockerfile}]
FROM python:3.11-slim

# Update, install curl for health checks, and clean up
RUN apt-get update && apt-get install -y curl && rm -rf /var/lib/apt/lists/*

WORKDIR /app

# Copy and install dependencies
COPY requirements.txt requirements.txt
RUN pip install --no-cache-dir -r requirements.txt

# Copy application code
COPY app.py app.py
COPY src src

# Expose the port the app runs on
EXPOSE 80

# Define the command to run the application
CMD ["python", "-m", "app"]

# Define the health check
HEALTHCHECK --interval=30s --timeout=5s --start-period=10s --retries=3 \
    CMD curl -f http://localhost/health || exit 1
\end{lstlisting}


\lstset{style=jsonstyle}


\begin{lstlisting}[language=json, caption={ECS Task Definition}, label={lst:json}]
{
    "compatibilities": [
        "EC2",
        "FARGATE"
    ],
    "containerDefinitions": [
        {
            "command": [
                "-m",
                "app"
            ],
            "cpu": 0,
            "entryPoint": [
                "python"
            ],
            "environment": [
                {
                    "name": "...",
                    "value": "..."
                },
                {
                    "name": "...",
                    "value": ..."
                }
            ],
            "essential": true,
            "healthCheck": {
                "command": [
                    "CMD-SHELL",
                    "curl -f http://localhost/health || exit 1"
                ],
                "interval": 300,
                "retries": 4,
                "startPeriod": 20,
                "timeout": 60
            },
            "image": "...amazonaws.com/mibotv6:production",
            "logConfiguration": {
                "logDriver": "awslogs",
                "options": {
                    "awslogs-group": "/ecs/mibot-v6-task",
                    "awslogs-create-group": "true",
                    "awslogs-region": "us-east-2",
                    "awslogs-stream-prefix": "ecs"
                }
            },
            "name": "mibot-v6-container",
            "portMappings": [
                {
                    "appProtocol": "http",
                    "containerPort": 80,
                    "hostPort": 80,
                    "protocol": "tcp"
                }
            ]
        }
    ],
    "cpu": "1024",
    "family": "mibot-v6-task",
    "memory": "4096",
    "networkMode": "awsvpc",
    "revision": 87,
    "status": "ACTIVE"
}
\end{lstlisting}



\begin{table}[ht!]
	\centering


	\begin{tabular}{ll}
		\toprule
		\textbf{Resource Path}            & \textbf{Allowed HTTP Method} \\
		\midrule

		% Use \multirow for cells that span multiple rows.
		% The first argument is the number of rows to span.
		\multirow{2}{*}{/chat}            & POST                         \\
		                                  & OPTIONS                      \\
		\cmidrule(l){2-2} % Adds a small line to separate groups

		\multirow{2}{*}{/get\_transcript} & POST                         \\
		                                  & OPTIONS                      \\
		\cmidrule(l){2-2}

		\multirow{2}{*}{/health}          & GET                          \\
		                                  & OPTIONS                      \\
		\cmidrule(l){2-2}

		\multirow{2}{*}{/info}            & GET                          \\
		                                  & OPTIONS                      \\
		\cmidrule(l){2-2}

		\multirow{2}{*}{/s3\_upload}      & POST                         \\
		                                  & OPTIONS                      \\

		\bottomrule
	\end{tabular}
	\caption{API Gateway Resource and Method Configuration}
	\label{tab:api_gateway}
\end{table}
