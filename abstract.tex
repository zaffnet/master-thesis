Tobacco use is a leading cause of preventable death, yet many smokers lack access to effective cessation support like motivational interviewing (MI). This thesis presents the development and evaluation of a fully generative MI chatbot designed to help smokers move towards the decision to quit. The chatbot leverages large language models (LLMs) to provide empathetic, person-centered counseling, aiming to overcome barriers of cost, availability, and stigma associated with traditional therapy.

This work makes two primary contributions. First, it details the design and iterative refinement of the MI chatbot. Its effectiveness is assessed through an empirical study with human smokers, who provided qualitative feedback and self-reported their readiness to quit before and after interacting with the chatbot. Second, this thesis introduces a novel methodology for creating and validating *synthetic smokers*—LLM-powered agents that realistically simulate the demographic and behavioral characteristics of human smokers in MI conversations. A framework for evaluating the fidelity of these synthetic agents is presented, with results demonstrating their ability to approximate the conversational patterns of real smokers in controlled experiments.

Collectively, these contributions advance the field of AI-assisted mental health by offering a scalable, evidence-based tool for smoking cessation and by providing a new method for the low-cost, high-fidelity simulation of human subjects in behavioral research.