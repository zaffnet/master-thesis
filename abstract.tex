Tobacco use is responsible for 7 million preventable deaths each year. For a large population of smokers, a therapeutic approach called Motivational Interviewing (MI) has shown effectiveness, yet many lack access to this support. This research presents an accessible alternative in the form of an expert-informed, large language model (LLM)-based \textbf{chatbot} that provides MI to smokers through unscripted, natural conversation. Further, this work contributes towards the development of \textbf{synthetic smokers}---LLM-based agents that simulate the demographic and behavioural characteristics of real human smokers, and can be used to test the chatbot during development.

A study with 106 smokers showed that a single conversation with the chatbot yielded a statistically significant average increase of +1.7 in participants' confidence to quit at one-week follow-up. On a metric of perceived empathy, the chatbot scored higher than its predecessors, but significantly lower than typical human counsellors. Further, an automated analysis of the transcripts showed the chatbot adhered to MI in 98\% of the utterances (higher than human counsellors) and elicited a high proportion of motivational language (59\%) from participants, comparable to high-quality human-delivered counselling sessions.

The fidelity of the synthetic smokers was validated by installing participants' behavioural attributes into their ``doppelgängers''. The automated analysis of transcripts showed that the level of motivation in doppelgängers' language strongly correlated ($r_{s}=0.57, p<0.0001$) with their twin participants. Further, synthetic smokers modulated their behaviour according to counselling quality, producing 54\% less motivational language when interacting with a \emph{confrontational} chatbot as compared to an \emph{MI-adherent} one. This shows their potential as a tool for testing automated counselling systems and training MI counsellors.