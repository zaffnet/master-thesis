Tobacco use is responsible for seven million preventable deaths each year. For the large population of smokers for whom a therapeutic approach called Motivational Interviewing (MI) has been effective, many lack access to this support. This research presents an accessible alternative: an expert-informed, large language model (LLM)-based \textbf{chatbot} that provides MI to smokers through unscripted, natural conversation. Additionally, this work contributes to the development of \textbf{synthetic smokers}---LLM-based agents that simulate the demographic and behavioural characteristics of real human smokers, which can be used to test the chatbot during development.

A study with 106 smokers showed that a single conversation with the chatbot yielded a statistically significant average increase of +1.7 in participants' confidence to quit at a one-week follow-up. On a metric of perceived empathy, the chatbot scored higher than its predecessors but significantly lower than typical human counsellors. An automated analysis of the transcripts showed the chatbot adhered to MI in 98\% of its utterances (a higher adherence rate than human counsellors) and elicited a high proportion of motivational language (59\%) from participants, comparable to high-quality, human-delivered counselling sessions.

The fidelity of the synthetic smokers was validated by installing participants' behavioural attributes into their "doppelgängers." An automated analysis of transcripts showed that the level of motivation in the doppelgängers' language strongly correlated ($r_{s}=0.57, p<0.0001$) with that of their twin participants. In addition, synthetic smokers modulated their behaviour according to the quality of the counselling, producing 54\% less motivational language when interacting with a \emph{confrontational} chatbot compared to an \emph{MI-adherent} one. This shows their potential as a tool for testing automated counselling systems and for training MI counsellors.
