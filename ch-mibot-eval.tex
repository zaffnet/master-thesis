\chapter{Evaluation of MIBot: Results from the Human Feasibility Study}
\label{ch:mibot-eval}

This chapter presents a thorough evaluation of MIBot's effectiveness through a feasibility study with 106 smokers. Building on the system design and implementation described in previous chapters, we assess MIBot's performance across four critical dimensions established in the smoking cessation and motivational interviewing literature: behavioural change readiness, perceived therapeutic empathy, adherence to MI principles, and elicitation of client change talk.

The chapter is organized to progress from primary outcomes to measurements of therapeutic process and, finally, to behavioural changes and user experiences. First, we report the primary outcome of changes in readiness to quit smoking (\Cref{sec:primary-outcome}). We then analyse how the chatbot performed on perceived empathy—measured using the CARE scale—and compare its performance to that of human healthcare professionals (\Cref{sec:perceived-empathy}). Next, we examine MIBot's adherence to MI principles through AutoMISC analysis and evaluate whether MIBot could maintain fidelity to therapeutic standards while successfully eliciting change talk from clients (\Cref{sec:mi-adherence}).

Following these core metrics, we investigate behavioural outcomes, including quit attempts and self-reported changes; analyse conversation flow, both quantitatively and qualitatively (\Cref{sec:conversation-dynamics}); and present illustrative case studies and sample outliers. Finally, we analyse participant feedback (\Cref{sec:feedback}) and discuss broader implications of our findings (\Cref{sec:synthesis}).

\section{Primary Outcome: Readiness to Quit}
\label{sec:primary-outcome}

\subsection*{Overall Changes in Readiness Rulers}

\Cref{table:mibot_ruler_summary} summarizes the mean (and standard deviation) of each readiness ruler before the conversation, immediately afterwards, and one week later. The table includes the change between the pre-conversation metric and one week later ($\Delta$). The Wilcoxon signed-rank test was applied to assess the significance of the change. Participants' confidence increased markedly from a baseline mean of 2.8 to 4.5 one week later ($\Delta=1.66, p<10^{-9}$). This represents an approximately 59\% relative improvement and constitutes our primary outcome measure, aligning with MI theory that confidence (self-efficacy) is a major predictor of behaviour change~\citep{Gwaltney2009-wj,Abar2013}. Importance also increased, albeit more modestly ($\Delta=0.5, p<0.005$), while readiness exhibited a small, non-significant change $(\Delta=0.23, p=0.22)$.

\begin{table}[ht!]
	\centering
	\small
	\setlength{\tabcolsep}{4pt}
	\renewcommand{\arraystretch}{1.1}
	\begin{tabular}{@{}lcccc@{}}
		\toprule
		\textbf{Ruler} & \textbf{Before}    & \textbf{After}     & \textbf{One Week}  & \textbf{$\Delta$}  \\
		               & \textbf{mean (SD)} & \textbf{mean (SD)} & \textbf{mean (SD)} & \textbf{mean (SD)} \\
		\midrule
		Importance     & 5.7 (2.6)          & 6.3 (2.9)          & 6.1 (2.7)          & 0.5 (1.7)**        \\
		Confidence     & 2.8 (2.0)          & 4.6 (2.6)          & 4.5 (2.7)          & 1.7 (2.4)***       \\
		Readiness      & 5.2 (2.8)          & 5.9 (2.8)          & 5.5 (3.0)          & 0.3 (2.4)          \\
		\bottomrule
	\end{tabular}
	\caption[MIBot Readiness Ruler Summary]{Means and standard deviations (SD) of readiness rulers (0--10 scale) for importance, confidence, and readiness, measured at three time points: before the conversation, immediately after, and one week later. The table also shows the mean change ($\Delta$) between the pre-conversation measurement and the one-week follow-up. Statistical significance of the change is assessed using the Wilcoxon signed-rank test (*** $p < 0.001$, ** $p < 0.01$, * $p < 0.05$).}
	\label{table:mibot_ruler_summary}
\end{table}

\Cref{fig:confidence_change_distribution} illustrates the distribution of one-week changes in confidence. Of the 106 participants, 64 (60.4\%) showed improvement, 21 (19.8\%) remained unchanged, and 21 (19.8\%) experienced a decrease. The median change was +1.0 point, with the interquartile range spanning 0 to +3 points. Notably, 15 participants (14.2\%) achieved gains of 5 or more points, representing substantial movement toward quitting confidence.

\begin{figure}[ht]
	\centering
	\includegraphics[width=0.8\linewidth]{fig/2024-11-14-MIV6.3A-2024-11-22-MIV6.3A_ruler_deltas_delta_with_week_later_keep_high_conf_False_change.png}
	\caption[Distribution of Confidence Changes]{Distribution of one-week changes in confidence scores (post-conversation minus pre-conversation). The x-axis represents the change in confidence on a 0-10 scale, and the y-axis represents the number of participants. The majority of participants (60.4\%) showed an improvement in confidence.}
	\label{fig:confidence_change_distribution}
\end{figure}



\subsection*{Stratified Analysis by Baseline Characteristics}

To understand which participants benefited most from MIBot, we stratified outcomes by baseline characteristics. \Cref{tab:baseline_confidence} shows that those starting with the lowest self-confidence ($n=31$, confidence $\leq 1$) experienced the largest improvement (+2.2 points), whereas participants with moderate or higher confidence gained approximately one and a half points. The three participants who began with very high confidence showed a decline, likely reflecting regression to the mean. Baseline confidence correlated negatively with the change (Spearman $r=-0.21$, $p<0.05$), indicating that MIBot is most beneficial for participants who are least confident in their ability to quit.

\begin{table}[ht!]
	\centering
	\small
	\renewcommand{\arraystretch}{1.1}
	\begin{tabular*}{\linewidth}{@{\extracolsep{\fill}}lccccc@{}}
		\toprule
		\textbf{Baseline} & \textbf{Sample} & \textbf{Baseline} & \textbf{Post-} & \textbf{1-week} & \textbf{Change from} \\
		\textbf{confidence range} & \textbf{size} & \textbf{confidence} & \textbf{conversation} & \textbf{follow-up} & \textbf{baseline} \\
		& & \textbf{mean (SD)} & \textbf{mean} & \textbf{mean (SD)} & \textbf{mean (SD)} \\
		\midrule
		0--1   & 31 & 0.5 (0.5) & 2.7 & 2.7 (2.5) & +2.2 (2.4)*** \\
		2--3   & 32 & 2.5 (0.5) & 2.0 & 4.2 (2.6) & +1.7 (2.5)** \\
		4--5   & 40 & 4.3 (0.5) & 1.4 & 5.8 (2.0) & +1.5 (2.1)*** \\
		$\geq$6 & 3 & 9.3 (0.6) & 0.6 & 7.7 (3.2) & $-1.7$ (2.9) \\
		\bottomrule
	\end{tabular*}
	\caption[Confidence Changes by Baseline Confidence]{Longitudinal changes in self-reported confidence scores (0--10 scale), stratified by baseline confidence level. The table shows the mean confidence scores at baseline, immediately post-conversation, and at 1-week follow-up, for different subgroups of participants based on their initial confidence. The change from baseline to 1-week follow-up is also presented. Statistical significance is determined by the Wilcoxon signed-rank test.}
	\label{tab:baseline_confidence}
\end{table}




Confidence changes varied across different smoking characteristics and quit history profiles. Participants were grouped by HSI (low 0--1, moderate 2--3, high 4--5, very high $\geq$6), daily cigarette consumption ($<$5, 5--9, 10--19, $\geq$20), whether they had made a quit attempt in the week before the study, and the number of prior attempts (0, 1--2, $\geq$3). The mean change in confidence for each subgroup is summarized in \Cref{tab:hsi_prequit}.

\begin{table}[ht!]
	\centering
	\small
	\renewcommand{\arraystretch}{1.1}
	\begin{tabular*}{\linewidth}{@{\extracolsep{\fill}}lccccc@{}}
		\toprule
		\textbf{Participant} & \textbf{Sample} & \textbf{Baseline} & \textbf{Post-} & \textbf{1-week} & \textbf{Change from} \\
		\textbf{characteristics} & \textbf{size} & \textbf{confidence} & \textbf{conversation} & \textbf{follow-up} & \textbf{baseline} \\
		& & \textbf{mean (SD)} & \textbf{mean (SD)} & \textbf{mean (SD)} & \textbf{mean (SD)} \\
		\midrule
		\multicolumn{6}{l}{\textit{Heaviness of Smoking Index}} \\
		\quad Low (0--1) & 28 & 3.5 (1.9) & 5.5 (2.3) & 4.9 (2.5) & +1.5 (2.7)** \\
		\quad Moderate (2--3) & 55 & 2.7 (2.0) & 4.1 (2.6) & 4.7 (2.9) & +2.0 (2.3)*** \\
		\quad High (4--5) & 21 & 2.5 (1.8) & 4.5 (2.3) & 3.3 (2.4) & +0.8 (2.1) \\
		\quad Very high ($\geq$6) & 2 & 1.5 (2.1) & 4.0 (5.7) & 5.0 (2.8) & +3.5 (0.7) \\
		\midrule
		\multicolumn{6}{l}{\textit{Daily cigarette consumption}} \\
		\quad $<$5 & 5 & 3.2 (1.3) & 5.2 (2.0) & 4.2 (2.6) & +1.0 (1.6) \\
		\quad 5--9 & 32 & 3.5 (2.5) & 5.2 (2.7) & 5.2 (3.0) & +1.7 (2.9)** \\
		\quad 10--19 & 38 & 2.8 (1.5) & 4.2 (2.5) & 4.5 (2.8) & +1.7 (2.1)*** \\
		\quad $\geq$20 & 31 & 2.1 (1.6) & 4.2 (2.5) & 3.7 (2.3) & +1.7 (2.3)*** \\
		\midrule
		\multicolumn{6}{l}{\textit{Pre-conversation quit attempt}} \\
		\quad Yes & 34 & 3.1 (1.6) & 5.1 (2.7) & 5.7 (2.6) & +2.6 (2.2)*** \\
		\quad No & 72 & 2.7 (2.1) & 4.3 (2.5) & 3.9 (2.6) & +1.2 (2.3)*** \\
		\midrule
		\multicolumn{6}{l}{\textit{Number of prior attempts}} \\
		\quad 0 & 72 & 2.7 (2.1) & 4.3 (2.5) & 3.9 (2.6) & +1.2 (2.3)*** \\
		\quad 1--2 & 16 & 3.3 (1.6) & 6.1 (2.5) & 6.9 (2.4) & +3.6 (2.0)*** \\
		\quad $\geq$3 & 18 & 2.8 (1.7) & 4.3 (2.6) & 4.7 (2.4) & +1.8 (2.0)** \\
		\bottomrule
	\end{tabular*}
	\caption[Confidence Changes by Smoking Characteristics]{Longitudinal changes in quit confidence scores (0--10 scale), stratified by baseline smoking characteristics and quit history. The table shows mean confidence scores at baseline, post-conversation, and 1-week follow-up for subgroups based on Heaviness of Smoking Index (HSI), daily cigarette consumption, pre-conversation quit attempts, and number of prior attempts. The change from baseline is also shown. Statistical significance is determined by the Wilcoxon signed-rank test.}
	\label{tab:hsi_prequit}
\end{table}


Participants with moderate nicotine dependence (HSI 2--3) showed the greatest gains (+2.0), compared with those with high dependence  (+0.8). Daily consumption showed little systematic difference across groups. Larger improvements were observed among participants reporting a conscious quit attempt in the week before the conversation ($n=34$), who showed confidence increases of +2.6, compared with +1.2 among those with no recent attempts. Among those with one or two previous attempts ($n=16$), the improvement was +3.6. These patterns could suggest that participants who were already contemplating change benefited the most.

\subsection*{Demographic Patterns}

\begin{table}[ht!]
	\centering
	\small
	\renewcommand{\arraystretch}{1.1}
	\begin{tabular*}{\linewidth}{@{\extracolsep{\fill}}lccccc@{}}
		\toprule
		\textbf{Demographic} & \textbf{Sample} & \textbf{Baseline} & \textbf{Post-} & \textbf{1-week} & \textbf{Change from} \\
		\textbf{characteristics} & \textbf{size} & \textbf{confidence} & \textbf{conversation} & \textbf{follow-up} & \textbf{baseline} \\
		& & \textbf{mean (SD)} & \textbf{mean (SD)} & \textbf{mean (SD)} & \textbf{mean (SD)} \\
		\midrule
		\multicolumn{6}{l}{\textit{Sex}} \\
		\quad Female & 57 & 2.5 (2.1) & 4.4 (2.8) & 4.1 (2.9) & +1.7 (2.5)*** \\
		\quad Male & 49 & 3.2 (1.7) & 4.7 (2.2) & 4.9 (2.5) & +1.7 (2.3)*** \\
		\midrule
		\multicolumn{6}{l}{\textit{Age}} \\
		\quad $<30$ years & 26 & 3.7 (2.1) & 5.5 (2.5) & 5.7 (2.7) & +1.9 (3.1)* \\
		\quad $\geq30$ years & 80 & 2.5 (1.8) & 4.3 (2.5) & 4.1 (2.6) & +1.6 (2.1)*** \\
		\midrule
		\multicolumn{6}{l}{\textit{Ethnicity}} \\
		\quad White & 80 & 2.7 (1.9) & 4.3 (2.6) & 4.0 (2.6) & +1.4 (2.2)*** \\
		\quad Other & 26 & 3.3 (2.0) & 5.3 (2.4) & 5.8 (2.8) & +2.5 (2.7)*** \\
		\midrule
		\multicolumn{6}{l}{\textit{Employment status}} \\
		\quad Full-time & 49 & 3.2 (1.9) & 4.8 (2.3) & 5.1 (2.6) & +1.9 (2.3)*** \\
		\quad Other & 57 & 2.5 (2.0) & 4.3 (2.8) & 3.9 (2.8) & +1.4 (2.4)*** \\
		\bottomrule
	\end{tabular*}
	\caption[Confidence Changes by Demographics]{Longitudinal changes in quit confidence scores (0--10 scale), stratified by demographic characteristics. The table shows mean confidence scores at baseline, post-conversation, and 1-week follow-up for subgroups based on sex, age, ethnicity, and employment status. The change from baseline is also shown. Statistical significance is determined by the Wilcoxon signed-rank test.}
	\label{table:demographics_wise_conf}
\end{table}

Demographic stratification reveals several patterns that warrant careful interpretation. Younger participants ($<30$ years) had higher baseline confidence (3.7, SD 2.1) than older groups (2.5, SD 1.8) and showed numerically larger improvements (mean +1.9, SD 3.1) compared to older participants (mean +1.6, SD 2.1). While baseline confidence differed by sex (2.5 for females vs 3.2 for males), week-later changes were identical (1.7). Participants identifying as non-white ethnicities had higher baseline confidence than white participants (3.3 vs 2.7) and showed larger gains (2.5 vs 1.4). These preliminary findings must be interpreted cautiously, as the study was not powered for subgroup analyses and baseline demographic differences were not statistically controlled for. See \Cref{table:demographics_wise_conf} for a summary of confidence changes by demographic characteristics.


\section{Perceived Empathy: CARE Scale Assessment}
\label{sec:perceived-empathy}

The Consultation and Relational Empathy (CARE) scale~\citep{10.1093/fampra/cmh621} measures patients' perceptions of their healthcare provider's empathy through 10 questions rated from 1 (poor) to 5 (excellent) (see \Cref{app:care-survey} for the full instrument). MIBot achieved a mean total score of 42 out of 50. To contextualize this performance, \Cref{table:care_comparison} compares MIBot's scores with human healthcare professionals from \citet{Bikker2015}.

\begin{table}[ht]
	\centering
	\small
	\setlength{\tabcolsep}{4pt}
	\renewcommand{\arraystretch}{1.1}
	\begin{tabular}{@{}lcc@{}}
		\toprule
		\textbf{Provider}                  & \textbf{Mean CARE} & \textbf{\% Perfect scores} \\
		\midrule
		MIBot                              & 42                 & 11                         \\
		Human healthcare professionals$^*$ & 46                 & 48                         \\
		\bottomrule
	\end{tabular}
        \caption[MIBot vs. Human CARE Scores]{Comparison of average CARE (Consultation and Relational Empathy) scores and the percentage of perfect scores between MIBot and typical human healthcare professionals. The data for human professionals are from~\cite{Bikker2015}.}
	\label{table:care_comparison}
\end{table}

While MIBot's mean score approaches that of human providers, the percentage of MIBot interactions achieving perfect scores (11\%) remains well below human benchmarks (48\%).

\begin{landscape}
	\begin{table}[htbp]
		\centering
		\footnotesize  % Even smaller than \small
		\begin{adjustbox}{scale=0.8, center}
			\begin{threeparttable}
				\setlength{\tabcolsep}{3pt}  % Reduce column spacing (default is 6pt)
				\renewcommand{\arraystretch}{0.9}  % Reduce row height
				\begin{tabular}{@{}l@{\hspace{3pt}}c|c|c|c|c|c|c|c|c|c@{\hspace{3pt}}c@{}}  % Tighter spacing
					\multicolumn{1}{l}{Characteristic}                                                    &
					\rotatebox{45}{\scriptsize making you feel at ease}                                   &
					\rotatebox{45}{\scriptsize letting you tell your story}                               &
					\rotatebox{45}{\scriptsize really listening}                                          &
					\rotatebox{45}{\scriptsize \parbox{2.7cm}{being interested in you as a whole person}} &
					\rotatebox{45}{\scriptsize \parbox{2.7cm}{fully understanding your concerns}}         &
					\rotatebox{45}{\scriptsize showing care and compassion}                               &
					\rotatebox{45}{\scriptsize being positive}                                            &
					\rotatebox{45}{\scriptsize explaining things clearly}                                 &
					\rotatebox{45}{\scriptsize helping you take control}                                  &
					\rotatebox{45}{\scriptsize \parbox{2.7cm}{making a plan of action with you}}          &
					\scriptsize \textbf{CARE}                                                                                                                                                                                                                                                                                                 \\
					\midrule
					\multicolumn{12}{@{}l}{\textbf{\scriptsize DEMOGRAPHIC CHARACTERISTICS}}                                                                                                                                                                                                                                                  \\
					\multicolumn{12}{@{}l}{\textit{\scriptsize Sex}}                                                                                                                                                                                                                                                                          \\
					Female (n=57)                                                                         & 4.6 (0.7) & 4.6 (0.7)                    & 4.5 (0.8)                    & \textbf{4.2 (0.9)$^\dagger$} & 4.4 (0.8) & 4.4 (0.9)                    & 4.6 (0.7) & 4.1 (1.1) & 3.8 (1.3) & 3.7 (1.5)                    & 42.8 (6.4) \\
					Male (n=49)                                                                           & 4.3 (1.0) & 4.5 (0.9)                    & 4.3 (1.0)                    & 3.7 (1.3)                    & 4.1 (1.0) & 4.2 (1.0)                    & 4.6 (0.8) & 4.1 (0.9) & 3.9 (1.3) & 3.4 (1.6)                    & 41.1 (8.6) \\
					p\textsuperscript{a}                                                                  & .174      & .844                         & .409                         & .026*                        & .078      & .470                         & .738      & .580      & .438      & .376                         & .489       \\[0.5pt]
					\addlinespace[1pt]
					\multicolumn{12}{@{}l}{\textit{\scriptsize Age}}                                                                                                                                                                                                                                                                          \\
					$<$30 (n=26)                                                                          & 4.4 (1.0) & 4.6 (0.7)                    & 4.2 (1.2)                    & 3.9 (1.2)                    & 4.0 (1.3) & 4.1 (1.0)                    & 4.5 (0.7) & 4.2 (1.0) & 4.1 (1.1) & 4.0 (1.3)                    & 42.0 (8.3) \\
					$\geq$30 (n=80)                                                                       & 4.5 (0.8) & 4.5 (0.8)                    & 4.5 (0.8)                    & 4.0 (1.1)                    & 4.3 (0.8) & 4.4 (0.9)                    & 4.6 (0.7) & 4.1 (1.0) & 3.7 (1.4) & 3.5 (1.6)                    & 42.0 (7.2) \\
					p\textsuperscript{a}                                                                  & .936      & .549                         & .267                         & .536                         & .454      & .164                         & .334      & .869      & .277      & .079                         & .788       \\[0.5pt]
					\addlinespace[1pt]
					\multicolumn{12}{@{}l}{\textit{\scriptsize Ethnicity}}                                                                                                                                                                                                                                                                    \\
					White (n=80)                                                                          & 4.5 (0.9) & 4.5 (0.8)                    & 4.4 (0.9)                    & 4.0 (1.2)                    & 4.2 (0.9) & 4.3 (1.0)                    & 4.6 (0.7) & 4.1 (1.0) & 3.7 (1.4) & 3.5 (1.5)                    & 41.9 (7.6) \\
					Other (n=26)                                                                          & 4.3 (0.8) & 4.5 (0.7)                    & 4.4 (0.9)                    & 4.0 (1.0)                    & 4.2 (1.1) & 4.2 (0.8)                    & 4.4 (0.8) & 4.1 (1.1) & 4.2 (1.1) & 4.0 (1.4)                    & 42.3 (7.1) \\
					p\textsuperscript{a}                                                                  & .134      & .498                         & .738                         & .770                         & .949      & .254                         & .122      & .795      & .099      & .071                         & .933       \\[0.5pt]
					\addlinespace[1pt]
					\multicolumn{12}{@{}l}{\textit{\scriptsize Employment}}                                                                                                                                                                                                                                                                   \\
					Full-Time (n=49)                                                                      & 4.3 (1.0) & 4.3 (0.9)                    & 4.2 (1.1)                    & 3.8 (1.3)                    & 4.1 (1.0) & 4.1 (1.1)                    & 4.5 (0.8) & 4.1 (1.0) & 3.8 (1.3) & 3.6 (1.5)                    & 40.7 (8.8) \\
					Other (n=57)                                                                          & 4.6 (0.7) & \textbf{4.7 (0.7)$^\dagger$} & \textbf{4.6 (0.8)$^\dagger$} & 4.2 (0.8)                    & 4.4 (0.9) & \textbf{4.5 (0.8)$^\dagger$} & 4.7 (0.6) & 4.2 (1.1) & 3.8 (1.4) & 3.6 (1.5)                    & 43.1 (6.0) \\
					p\textsuperscript{a}                                                                  & .174      & .013*                        & .028*                        & .144                         & .152      & .012*                        & .296      & .521      & .850      & .924                         & .263       \\[0.5pt]
					\midrule
					\multicolumn{12}{@{}l}{\textbf{\scriptsize BEHAVIOURAL CHARACTERISTICS}}                                                                                                                                                                                                                                                  \\
					\addlinespace[1pt]
					\multicolumn{12}{@{}l}{\textit{\scriptsize Confidence}}                                                                                                                                                                                                                                                                   \\
					0-1 (n=31)                                                                            & 4.5 (0.9) & 4.5 (0.8)                    & 4.4 (0.9)                    & 3.9 (1.3)                    & 4.4 (0.8) & 4.4 (1.0)                    & 4.5 (0.8) & 4.1 (1.1) & 3.7 (1.4) & 3.3 (1.6)                    & 41.5 (7.7) \\
					2-3 (n=32)                                                                            & 4.5 (0.7) & 4.6 (0.8)                    & 4.6 (0.8)                    & 4.0 (1.0)                    & 4.2 (0.9) & 4.4 (0.9)                    & 4.7 (0.7) & 4.0 (1.1) & 3.8 (1.4) & 3.2 (1.5)                    & 42.1 (6.9) \\
					4-5 (n=40)                                                                            & 4.3 (1.0) & 4.5 (0.8)                    & 4.3 (1.1)                    & 4.0 (1.2)                    & 4.2 (1.1) & 4.2 (1.0)                    & 4.5 (0.7) & 4.2 (1.0) & 4.0 (1.3) & 4.0 (1.4)                    & 42.1 (8.2) \\
					$\geq$6 (n=3)                                                                         & 4.7 (0.6) & 5.0 (0.0)                    & 4.7 (0.6)                    & 4.7 (0.6)                    & 3.7 (0.6) & 4.7 (0.6)                    & 5.0 (0.0) & 4.3 (0.6) & 3.7 (0.6) & \textbf{4.3 (1.2)$^\dagger$} & 44.7 (0.6) \\
					p\textsuperscript{b}                                                                  & .534      & .471                         & .691                         & .621                         & .363      & .623                         & .248      & .962      & .798      & .032*                        & .922       \\[0.5pt]
					\addlinespace[1pt]
					\multicolumn{12}{@{}l}{\textit{\scriptsize HSI}}                                                                                                                                                                                                                                                                          \\
					Low (0-1) (n=28)                                                                      & 4.1 (1.1) & 4.3 (1.0)                    & 4.1 (1.2)                    & 4.0 (1.2)                    & 4.1 (1.2) & 4.0 (1.2)                    & 4.4 (0.9) & 3.9 (1.2) & 3.7 (1.3) & 3.8 (1.4)                    & 40.4 (9.1) \\
					Med. (2-3) (n=55)                                                                     & 4.5 (0.8) & 4.6 (0.7)                    & 4.5 (0.9)                    & 4.0 (1.1)                    & 4.3 (0.9) & 4.3 (0.9)                    & 4.6 (0.7) & 4.2 (0.9) & 4.0 (1.1) & 3.6 (1.5)                    & 42.5 (7.5) \\
					High (4-5) (n=21)                                                                     & 4.7 (0.5) & 4.7 (0.5)                    & 4.5 (0.7)                    & 4.0 (1.0)                    & 4.4 (0.8) & 4.6 (0.7)                    & 4.7 (0.5) & 4.4 (0.9) & 3.5 (1.7) & 3.3 (1.7)                    & 42.8 (4.8) \\
					V.high ($\geq$6) (n=2)                                                                & 5.0 (0.0) & 5.0 (0.0)                    & 5.0 (0.0)                    & 5.0 (0.0)                    & 4.5 (0.7) & 4.5 (0.7)                    & 5.0 (0.0) & 3.0 (1.4) & 3.5 (2.1) & 3.0 (2.8)                    & 43.5 (6.4) \\
					p\textsuperscript{b}                                                                  & .126      & .281                         & .169                         & .421                         & .890      & .351                         & .423      & .167      & .672      & .809                         & .678       \\[0.5pt]
					\addlinespace[1pt]
					\multicolumn{12}{@{}l}{\textit{\scriptsize Cigarettes/Day}}                                                                                                                                                                                                                                                               \\
					$<$5 (n=5)                                                                            & 4.2 (0.4) & 4.2 (0.8)                    & 4.2 (0.4)                    & 4.0 (0.7)                    & 5.0 (0.0) & 4.2 (0.8)                    & 4.8 (0.4) & 4.2 (0.8) & 4.2 (0.8) & 4.4 (0.9)                    & 43.4 (4.4) \\
					5-9 (n=32)                                                                            & 4.4 (0.9) & 4.5 (0.7)                    & 4.4 (0.9)                    & 4.1 (1.2)                    & 4.2 (0.9) & 4.3 (0.8)                    & 4.5 (0.7) & 4.2 (0.7) & 3.9 (1.0) & 3.9 (1.4)                    & 42.3 (6.8) \\
					10-19 (n=38)                                                                          & 4.3 (1.0) & 4.5 (0.9)                    & 4.4 (1.0)                    & 3.9 (1.1)                    & 4.2 (1.0) & 4.2 (1.2)                    & 4.6 (0.8) & 4.1 (1.1) & 3.8 (1.5) & 3.5 (1.5)                    & 41.6 (9.1) \\
					$\geq$20 (n=31)                                                                       & 4.6 (0.7) & 4.6 (0.7)                    & 4.3 (1.0)                    & 4.0 (1.2)                    & 4.2 (1.0) & 4.5 (0.8)                    & 4.7 (0.5) & 4.1 (1.2) & 3.8 (1.5) & 3.3 (1.7)                    & 42.0 (6.6) \\
					p\textsuperscript{b}                                                                  & .199      & .550                         & .453                         & .932                         & .176      & .499                         & .521      & .959      & .916      & .327                         & .960       \\[0.5pt]
					\addlinespace[1pt]
					\multicolumn{12}{@{}l}{\textit{\scriptsize Quit Attempts}}                                                                                                                                                                                                                                                                \\
					0 (n=72)                                                                              & 4.4 (0.9) & 4.5 (0.8)                    & 4.4 (1.0)                    & 3.9 (1.2)                    & 4.2 (1.0) & 4.2 (1.0)                    & 4.5 (0.7) & 4.1 (1.1) & 3.7 (1.4) & 3.5 (1.5)                    & 41.4 (7.9) \\
					1-2 (n=16)                                                                            & 4.6 (0.6) & 4.6 (0.8)                    & 4.6 (0.6)                    & 4.1 (0.9)                    & 4.6 (0.5) & 4.4 (1.0)                    & 4.6 (0.6) & 4.4 (0.9) & 3.9 (1.1) & 4.2 (1.2)                    & 44.1 (5.4) \\
					$\geq$3 (n=18)                                                                        & 4.4 (1.0) & 4.7 (0.8)                    & 4.3 (1.0)                    & 4.2 (1.1)                    & 4.2 (0.9) & 4.4 (0.8)                    & 4.7 (0.8) & 4.1 (0.8) & 4.1 (1.2) & 3.6 (1.7)                    & 42.7 (7.5) \\
					p\textsuperscript{b}                                                                  & .878      & .279                         & .709                         & .583                         & .265      & .564                         & .569      & .523      & .696      & .220                         & .500       \\[1pt]
					\bottomrule
				\end{tabular}
				\begin{tablenotes}[para,flushleft]  % `para' makes notes in paragraph form to save space
					\tiny  % Tiny font for notes
                                        \item Mean (SD). $^\dagger$ Higher mean (significant). \textsuperscript{a} Mann–Whitney; \textsuperscript{b} Kruskal–Wallis. ***$p<0.001$, **$p<0.01$, *$p<0.05$
				\end{tablenotes}
			\end{threeparttable}
		\end{adjustbox} % END of the wrapper
		\caption[CARE Scores by Demographics and Behaviour]{CARE (Consultation and Relational Empathy) scale scores, broken down by various demographic and behavioural characteristics of the participants. The table shows the mean and standard deviation of scores for each of the 10 CARE items, as well as the total CARE score.}
		\label{tab:care_comprehensive}
	\end{table}
\end{landscape}


\subsection*{Question-by-Question Analysis of CARE Survey}
\begin{table}[ht]
	\centering
	\small
	\setlength{\tabcolsep}{3pt}
	\renewcommand{\arraystretch}{1.1}
	\begin{tabular}{@{}lc@{}}
		\toprule
		\textbf{How was MIBot at...}              & \textbf{Mean (SD)} \\
		\midrule
		Being positive                            & 4.6 (0.7)          \\
		Letting you tell your ``story''           & 4.5 (0.8)          \\
		Making you feel at ease                   & 4.4 (0.9)          \\
		Really listening                          & 4.4 (0.9)          \\
		Showing care and compassion               & 4.3 (0.9)          \\
		Fully understanding your concerns         & 4.2 (1.0)          \\
		Explaining things clearly                 & 4.1 (1.0)          \\
		Being interested in you as a whole person & 4.0 (1.1)          \\
		Helping you take control                  & 3.8 (1.3)          \\
		Making a plan of action with you          & 3.6 (1.5)          \\
		\midrule
		\textbf{Total Score}                      & 42.2 (7.5)         \\
		\bottomrule
	\end{tabular}
	\caption[Mean CARE Scores per Question]{Mean scores for each of the 10 questions on the CARE (Consultation and Relational Empathy) scale, rated on a 1-5 scale. The table is sorted from the highest-scoring to the lowest-scoring question.}
	\label{table:care_question_means}
\end{table}
\Cref{table:care_question_means} presents the mean scores for each CARE question, revealing specific strengths and weaknesses in MIBot's empathic performance. The chatbot performed best on `being positive' (4.6, SD 0.7) and `letting you tell your ``story''' (4.5, SD 0.8), while it scored lowest on `helping you take control' (3.8, SD 1.3) and `making a plan of action with you' (3.6, SD 1.5).


The poor performance on some questions may be due to the chatbot's lack of emotional intelligence~\citep{sabour-etal-2024-emobench} or collaboration skills~\citep{yang-etal-2024-human}. \Cref{tab:care_comprehensive} presents CARE scale scores across demographic and behavioural characteristics. Female participants rated MIBot markedly higher than males on ``being interested in you as a whole person'' (4.2 vs 3.7, p=.026). Likewise, participants in non-full-time employment gave markedly higher scores than full-time workers on three dimensions: `letting you tell your story' (4.7 vs 4.3, p=.013), `really listening' (4.6 vs 4.2, p=.028), and `showing care and compassion' (4.5 vs 4.1, p=.012). No significant differences were found across age, ethnicity, smoking intensity, or quit attempt categories.







\section{Comparing Fully Generative MIBot v6.3 With Partially Scripted MIBot v5.2}
\label{sec:comparison-v52}

To contextualize the performance of our fully generative approach, we compare MIBot v6.3A with its predecessor, MIBot v5.2, a hybrid system that combined scripted questions with LLM-generated reflections~\citep{brown2023mi}. MIBot v5.2 employed a hybrid approach using scripted open-ended questions followed by GPT-2 XL-generated MI-style reflections based on participants' responses to the questions.

\subsection*{Readiness Ruler Comparisons}
\begin{figure}[htbp]
	\centering
	\begin{subfigure}[b]{0.48\textwidth}
		\centering
		\includegraphics[width=\textwidth]{fig/MIV5.2_ruler_deltas_delta_with_week_later_keep_high_conf_False_change.png}
		\caption{MIBot v5.2}
		\label{fig:confidence_v5.2}
	\end{subfigure}
	\hfill
	\begin{subfigure}[b]{0.48\textwidth}
		\centering
		\includegraphics[width=\textwidth]{fig/2024-11-14-MIV6.3A-2024-11-22-MIV6.3A_ruler_deltas_delta_with_week_later_keep_high_conf_False_change.png}
		\caption{MIBot v6.3A}
		\label{fig:confidence_v6.3}
	\end{subfigure}


	\caption[Confidence Change Distributions for MIBot v5.2 and v6.3A]{Distribution of week-later confidence changes from baseline for MIBot v5.2 and MIBot v6.3A. The x-axis shows the change in confidence score, and the y-axis shows the number of participants. Red bars indicate a decrease in confidence, while blue bars indicate an increase or no change.}
	\label{fig:confidence_distributions}
\end{figure}










MIBot v5.2 achieved a mean confidence increase of 1.3 (SD 2.3, $p<0.001$)
from baseline to one week later among 100 participants, and
MIBot v6.3A achieved an increase of 1.7 (SD 2.4, $p<0.001$) among 106 participants.
This difference was not statistically significant ($t(203.60) = 0.98$,
$p = 0.165$, one-tailed, Cohen's $d = 0.14$). As shown in \Cref{fig:confidence_distributions},
the distribution of confidence changes reveals several patterns: MIBot v6.3A
resulted in more participants maintaining their baseline confidence (28\% vs.\ 23\%
with no change) and fewer experiencing decreased confidence (13\% vs.\ 17\%).

Regarding importance to quit, MIBot v5.2 achieved a significant increase of 0.7 points (SD 2.0, $p<0.001$), while MIBot v6.3A showed a more modest yet still significant gain of 0.5 points (SD 1.7, $p<0.005$). Readiness changes were minimal and non-significant for both versions (v5.2: 0.4 points, SD 1.7, $p=0.01$; v6.3A: 0.3 points, SD 2.4, $p=0.22$).

\subsection*{Perceived Empathy Comparisons}

\begin{figure}[htbp]
	\centering
	\begin{subfigure}[b]{0.48\textwidth}
		\centering
		\includegraphics[width=\textwidth]{fig/MIV5.2_care_scores_histogram.png}
		\caption{MIBot v5.2 (hybrid)}
		\label{fig:care_v5.2}
	\end{subfigure}
	\hfill
	\begin{subfigure}[b]{0.48\textwidth}
		\centering
		\includegraphics[width=\textwidth]{fig/2024-11-14-MIV6.3A-2024-11-22-MIV6.3A_care_scores_histogram.png}
		\caption{MIBot v6.3A (fully generative)}
		\label{fig:care_v6.3}
	\end{subfigure}
	\caption[CARE Score Distributions for MIBot v5.2 and v6.3A]{Distribution of CARE (Consultation and Relational Empathy) empathy scores for MIBot v5.2 and MIBot v6.3A. The x-axis represents the total CARE score, and the y-axis represents the number of participants. The distribution for the fully generative v6.3A is shifted to the right, indicating higher perceived empathy.}
	\label{fig:care_distributions}
\end{figure}

\begin{figure}[htbp]
	\centering
	\includegraphics[width=0.98\textwidth]{fig/combined_care_scores_with_improvement.png}
	\caption[Comparison of Mean CARE Scores per Question]{Comparison of mean CARE (Consultation and Relational Empathy) scores for each question between MIBot v5.2 (hybrid) and v6.3A (fully generative). The x-axis lists the 10 CARE questions, and the y-axis shows the mean score. The fully generative version shows improvement across all dimensions.}
	\label{fig:care_questions}
\end{figure}

MIBot v5.2 achieved a mean CARE score of 36 (SD 9.1), with only 3\% of participants awarding a perfect score, while MIBot v6.3A achieved a mean score of 42 (SD 7.5), with 11\% receiving perfect scores. This difference was statistically significant ($t(191.67) = 4.96$, $p < 0.001$, Cohen's $d = 0.70$), representing a medium-to-large effect size and suggesting that the fully generative responses of v6.3A markedly improved perceived empathy. \Cref{fig:care_distributions} illustrates the distributional shift between versions. The hybrid v5.2 shows a relatively normal distribution centred in the mid-30s, with considerable spread across all score ranges, whereas v6.3A demonstrates a clear rightward skew, with 40\% of participants rating the chatbot in the highest range (46--50).

To understand which aspects of empathetic interaction improved most, \Cref{fig:care_questions} presents the mean scores across all ten CARE dimensions. The fully generative approach exhibits improvements across every dimension, with particularly notable gains in emotional and communicative aspects. The largest relative improvement was observed in ``showing care and compassion'' (+32\%), reflecting the chatbot's improved ability to maintain an encouraging tone. The dimension of ``making a plan of action with you'' remained the weakest aspect for both versions (2.73 for v5.2, 3.59 for v6.3A), despite showing considerable relative improvement (+32\%). Notably, MIBot v6.3A was prompted with detailed guidelines on how to make a plan of action with the client. Despite this, planning remains one of its weaker areas.



\subsection*{Implications of the Comparison}

The fully generative MIBot v6.3A chatbot scored higher on CARE, a measure of perceived therapeutic alliance. This improvement suggests that participants experienced more authentic, personalized interactions when the entire conversation---not just reflections---emerged from the language model's contextual understanding. However, both chatbots scored similarly on our primary metric of effectiveness, namely the week-later change in confidence, indicating that the core therapeutic mechanism of MI may be responsible for most of the gains in confidence. This finding aligns with prior work showing that even simple question-asking can produce substantial benefits~\citep{brown2023mi}, though the improved empathy of full generation may improve engagement and retention in real-world deployment.

\section{AutoMISC Analysis}
\label{sec:mi-adherence}

To evaluate the chatbot's adherence to MI principles, we analysed counsellor and client utterances from the feasibility study transcripts (Chapter 4) using AutoMISC, an automated annotation system originally described in \citet{mahmood-etal-2025-fully}\footnote{A detailed description of the AutoMISC system by \citet{ali2025automated} is forthcoming}.
The system assigns behavioural codes to each utterance: counsellor utterances are classified as MI-Consistent (MICO), MI-Inconsistent (MIIN), Reflection (R), Question (Q), or Other (O), while client utterances are categorized as change talk (C), sustain talk (S), or neutral (N). Following annotation of all utterances, we computed per-transcript summary metrics to quantify MI adherence. These comprise:

\begin{itemize}

	\item \textbf{Percentage MI-Consistent Responses (\%MIC):} The proportion of counsellor utterances that align with MI principles. Higher values indicate greater adherence to MI methodology.

	\item \textbf{Reflection-to-Question Ratio (R:Q):} The ratio of counsellor utterances labelled as reflection (R) to those labelled as question (Q). This metric assesses the balance between reflective listening and questioning. Values between 1 and 2 are considered indicative of proficiency~\citep{moyers2016miti}.

	\item \textbf{Percentage Change Talk (\%CT):} The proportion of client utterances expressing motivation toward behaviour change. Higher values are associated with improved behavioural outcomes~\citep{Apodaca2009}.

\end{itemize}

\subsection{Contextualizing AutoMISC Metrics With the HLQC Dataset}
To provide a point of comparison for the MISC summary metrics, we also ran AutoMISC on the HighLowQualityCounselling (HLQC) dataset~\cite{perez-rosas-etal-2019-makes}, a publicly available\footnote{\url{https://lit.eecs.umich.edu/downloads.html}} corpus of transcribed MI counselling demonstrations. The HLQC dataset comprises 155 high-quality (HLQC\_\textbf{HI}) and 104 low-quality (HLQC\_\textbf{LO}) transcripts sourced from public websites.
We computed summary scores separately for these subsets and then compared MIBot's summary metrics against those of both HLQC\_HI and HLQC\_LO.

\subsection{Counsellor Behaviour Metrics}
\begin{table}[ht]
	\centering
	\small
	\setlength{\tabcolsep}{4pt}
	\renewcommand{\arraystretch}{1.1}
	\begin{tabular}{@{}lcc@{}}
		\toprule
		\textbf{Metric}                    & \textbf{MIBot} & \textbf{HLQC\_HI (high-quality)} \\
		\midrule
		\%MI-consistent (\%MIC)            & 98 (3.6)       & 92 (9.8)                         \\
		Reflection-to-Question Ratio (R:Q) & 1.3 (0.3)      & 2.3 (5.7)                        \\
		\bottomrule
	\end{tabular}
	\caption[AutoMISC Counsellor Metrics for MIBot vs. Human]{Counsellor-specific summary metrics from AutoMISC for MIBot, compared to high-quality human counselling sessions from the HLQC dataset. The table shows the mean (and standard deviation) for the percentage of MI-consistent responses (\%MIC) and the reflection-to-question ratio (R:Q).}
	\label{table:automisc_summary}
\end{table}

\Cref{fig:misc_distributions} shows violin plots comparing the distribution of \%MIC, R:Q, and \%CT across MIBot conversations with the HLQC benchmarks. The \%MIC scores are tightly clustered near 100\%, with only a few transcripts falling below 80\%. The R:Q distribution is centred around 1.3, showing less variance than human counsellors who range from pure question-asking (R:Q near 0) to heavy reflection use (R:Q $>$ 5).

\begin{figure}[htpb]
	\centering
	\begin{subfigure}[b]{0.9\textwidth}
		\centering
		\includegraphics[height=0.25\textheight, keepaspectratio]{fig/mic_enhanced.pdf}
		\caption{\%MIC}
	\end{subfigure}

	\begin{subfigure}[b]{0.9\textwidth}
		\centering
		\includegraphics[height=0.25\textheight, keepaspectratio]{fig/rq_enhanced.pdf}
		\caption{R:Q}
	\end{subfigure}

	\begin{subfigure}[b]{0.9\textwidth}
		\centering
		\includegraphics[height=0.25\textheight, keepaspectratio]{fig/cs_enhanced.pdf}
		\caption{\%CT}
	\end{subfigure}

	\caption[Comparison of MISC summary score distributions for MIBot and the HLQC dataset]{Comparison of MISC summary score distributions for MIBot and the HLQC dataset (high and low quality). The figure shows violin plots for (a) Percentage MI-Consistent Responses (\%MIC), (b) Reflection to Question Ratio (R:Q), and (c) Percentage Client Change Talk (\%CT). Taken from \citet{ali2025thesis}}.
	\label{fig:misc_distributions}
\end{figure}


\subsection{Client Change Talk Analysis}



\begin{table}[ht]
	\centering
	\small
	\setlength{\tabcolsep}{4pt}
	\renewcommand{\arraystretch}{1.1}
	\begin{tabular}{@{}lcc@{}}
		\toprule
		\textbf{Metric}      & \textbf{MIBot} & \textbf{HLQC\_HI (high-quality)} \\
		\midrule
		\%Change Talk (\%CT) & 59 (25.6)      & 53 (28.4)                       \\
		\bottomrule
	\end{tabular}
	\caption[AutoMISC Client Metrics for MIBot vs. Human]{Client-specific summary metric from AutoMISC for MIBot, compared to high-quality human counselling sessions from the HLQC dataset. The table shows the mean (and standard deviation) for the percentage of change talk (\%CT).}
	\label{table:automisc_summary_client}
\end{table}


\Cref{fig:misc_distributions}(c) illustrates the distribution of \%CT across conversations. Most sessions by MIBot had a \%CT of $>$ 60\%, and the distribution closely matches that of the \textbf{HLQC\_HI} dataset. Overall, the mean \%CT was 59\% for MIBot, compared to 53\% for the \textbf{HLQC\_HI} dataset.

\section{Behavioural Outcomes}


\label{sec:behavioural-outcomes}

\subsection*{Quit Attempts at One Week}

\begin{table}[ht]
	\centering
	\small
	\setlength{\tabcolsep}{4pt}
	\renewcommand{\arraystretch}{1.1}
	\begin{tabular}{@{}lcc@{}}
		\toprule
		\textbf{Quit Attempt Status} & \textbf{Pre-conversation} & \textbf{Post-conversation} \\
		\midrule
		Made attempt                 & 34 (32.1\%)               & 37 (34.9\%)                \\
		No attempt                   & 72 (67.9\%)               & 69 (65.1\%)                \\
		\midrule
		New attempters               & --                        & 15 (14.2\%)                \\
		Sustained attempters         & --                        & 22 (20.8\%)                \\
		\bottomrule
	\end{tabular}
	\caption[Quit Attempt Behaviour Before and After MIBot]{Summary of participants' quit attempt behaviour before and after the MIBot conversation. The table shows the number and percentage of participants who made a quit attempt in the week prior to the conversation and in the week following the conversation.}
	\label{table:quit_attempts}
\end{table}

Participants who made quit attempts post-conversation showed larger confidence gains (mean +2.24) compared to those who did not attempt (+1.49, $p=0.0664$). This bidirectional relationship---where confidence predicts attempts and attempts reinforce confidence---aligns with social cognitive theory~\citep{Bandura1986}.


\section{Conversation Analysis}
\label{sec:conversation-analysis}

\subsection*{Quantitative Flow}
\label{sec:conversation-dynamics}


Conversations ranged from 36 to 163 utterances (median 78; mean 80.71, SD 25.54), with counsellor utterances comprising 71.7\% of the total. The median conversation lasted approximately 30 minutes based on the lone participant self-report that mentioned duration. Table~\ref{table:conversation-dynamics} summarizes main quantitative metrics.

\begin{table}[ht]
\centering
\small
\setlength{\tabcolsep}{4pt}
\renewcommand{\arraystretch}{1.2}
\begin{tabular}{@{}lrr@{}}
\toprule
\textbf{Metric} & \textbf{Mean (SD)} & \textbf{Range} \\
\midrule
Total utterances               & 80.7 (25.5) & 36--163 \\
Counsellor utterances          & 57.9 (20.0) & 26--123 \\
Client utterances              & 22.8 (6.9)  & 9--40 \\
Words per counsellor utterance & 14.6 (1.5)  & 11--19 \\
Words per client utterance     & 7.0 (2.6)   & 1--13 \\
\bottomrule
\end{tabular}
	\caption[Conversation Dynamics between Participants and MIBot: Quantitative Metrics]{Quantitative metrics of the conversation dynamics between participants and MIBot. The table includes statistics on the total number of utterances, counsellor and client utterances, words per utterance, and session duration.}
	\label{table:conversation-dynamics}
\end{table}

Longer conversations correlated with better outcomes ($r = 0.29$ for confidence change), but the relationship was non-linear. Conversations under 60 utterances rarely produced substantial gains (only 21\% reached $\geq$2-point confidence increases), while those exceeding 130 utterances showed diminishing returns (50\% achieving $\geq$2-point gains with a small sample). This suggests an optimal engagement window of roughly 110--120 exchanges before improvements level off.


\subsection*{Qualitative Thematic Analysis}

To understand the qualitative aspects of the conversations, a thematic analysis was performed on the full corpus of transcripts. The analysis was conducted by two researchers who independently reviewed the transcripts to identify recurring patterns and themes. They then met to compare their findings, discuss discrepancies, and collaboratively develop a final set of themes that characterized successful therapeutic engagement. This process revealed three recurring patterns:




%%%%%%%%
\subsubsection{Stress and Coping Narratives}

56\% of conversations referenced this theme across 59 transcripts.
Participants described cigarettes as tools for emotion regulation and short-term relief. Discussions often focused on stress at work, family pressures, or using smoke breaks as a momentary escape.

\begin{quote}
\emph{"It is a stress reliever for me."} \\
\emph{"Every time I have tried to quit, other things in my life have come up and it helps me to calm down to deal with them."} \\
\emph{"My job becomes more stressful."}
\end{quote}

\subsubsection{Social and Ritualistic Aspects}

51\% of conversations referenced this theme across 54 transcripts.
Smoking was entwined with social identity and daily rituals. Clients highlighted smoke breaks with colleagues, shared routines with partners, and the fear of losing social contact if they quit.

\begin{quote}
\emph{"I'm embarrassed that I can't just quit by myself."} \\
\emph{"I feel like if I actually commit to making a change, it could be possible."} \\
\emph{"I could speak to my friends and family about seeing a counsellor to get their support and advice."}
\end{quote}

\subsubsection{Ambivalence Themes}

19\% of conversations referenced this theme across 20 transcripts.
Clients expressed motivational ambivalence, simultaneously acknowledging reasons to quit while defending the role of smoking in their lives. Conversations frequently normalized these mixed feelings before exploring change talk.

\begin{quote}
\emph{"My health and the appearance of my teeth are probably the main reasons I want to quit."} \\
\emph{"It makes me want to quit"} \\
\emph{"My wife wants me to quit and I want to quit for my health and my marriage."}
\end{quote}

\subsubsection{Success Stories}

Two clients reported marked increases in confidence within one week, moving from very low to high confidence after extended dialogues. Their narratives illustrate shifts in perspective:

\begin{quote}
\emph{"Seeing underage kids smoke has made me see it in a negative light."} \\
\emph{"Perhaps you tell me how you would wish to help me as you have stated above."}
\end{quote}

\subsubsection{Enjoyment-Focused Smokers}

Example quotes include:
    \begin{quote}
    \emph{"At night you can enjoy the cooler air and watch the stars and think about things."} \\
    \emph{"Frankly, I enjoy it, and long ago I tried hypnosis and it worked; I found it distasteful."}
    \end{quote}

%%%%%%%%%


\section{User Experience and Feedback}
\label{sec:feedback}

\subsection*{Post-Conversation Feedback}

Three open-ended questions captured immediate post-conversation reactions. Thematic analysis revealed distinct patterns in positive and negative responses (see \Cref{app-feedback} for a selection of verbatim responses).

\textbf{Positive Themes (92\% enjoyed the experience):}
\begin{itemize}
        \item Non-judgemental approach: ``Finally someone (something?) that didn't lecture me.''
	\item Structured discussion: ``Helped me organize my thoughts about quitting.''
	\item Convenience and privacy: ``Could be honest without embarrassment.''
	\item Surprising depth: ``More helpful than expected from a bot.''
\end{itemize}

\textbf{Negative Themes (34\% found it unhelpful):}
\begin{itemize}
	\item Lack of personalization: ``Felt like generic responses sometimes.''
	\item Missing human connection: ``Technically correct but emotionally flat.''
	\item Repetitiveness: ``Kept asking similar questions different ways.''
	\item Insufficient challenge: ``Too accepting, didn't push me enough.''
\end{itemize}

\subsection*{User Segmentation}

Based on feedback analysis, we derived two binary metrics: ``LikedBot'' (92\% positive) and ``FoundBotHelpful'' (66\% positive). The discrepancy suggests that while MIBot creates an engaging experience, translating engagement into perceived therapeutic value remains challenging.

Cross-tabulation revealed four user segments, as shown in \Cref{table:user_segments}.

\begin{table}[ht]
	\centering
	\small
	\setlength{\tabcolsep}{4pt}
	\renewcommand{\arraystretch}{1.1}
	\begin{tabular}{@{}lcc@{}}
		\toprule
		\textbf{Segment}        & \textbf{Liked \& Helpful} & \textbf{\% of Sample} \\
		\midrule
		Enthusiasts             & Yes \& Yes                & 61\%                  \\
		Entertained Sceptics    & Yes \& No                 & 31\%                  \\
		Reluctant Beneficiaries & No \& Yes                 & 5\%                   \\
		Dissatisfied            & No \& No                  & 3\%                   \\
		\bottomrule
	\end{tabular}
        \caption[Segmentation of MIBot Participants based on their Experiences]{Segmentation of users based on their reported enjoyment and perceived helpfulness of the MIBot conversation. The table shows the percentage of participants in each of the four segments: Enthusiasts, Entertained Sceptics, Reluctant Beneficiaries, and Dissatisfied.}
	\label{table:user_segments}
\end{table}

``Entertained Sceptics''---those who enjoyed but did not find it helpful---often wanted more directive advice or concrete tools rather than open-ended conversation.


\section{Discussion}
\label{sec:synthesis}



Multivariate regression analysis identified the strongest predictors of confidence improvement:

\begin{enumerate}
	\item \textbf{Low baseline confidence} ($\beta=-0.31$, $p<0.001$): greatest gains for those starting lowest.
	\item \textbf{Recent quit attempt} ($\beta=0.28$, $p<0.01$): prior action amplifies intervention effects.
	\item \textbf{Conversation length} ($\beta=0.21$, $p<0.05$): deeper engagement yields better outcomes.
	\item \textbf{Change talk ratio} ($\beta=0.18$, $p<0.05$): client language predicts behaviour change.
	\item \textbf{Age <40} ($\beta=0.15$, $p<0.05$): younger participants more responsive.
\end{enumerate}

Together, these factors explained 34\% of the variance in confidence change ($R^2=0.34$, $F(5,100)=10.32$, $p<0.001$), suggesting that ideal candidates are younger smokers with low confidence who have recently attempted to quit and are willing to engage in extended conversation.

\subsection*{Comparison With Literature Benchmarks}

MIBot's performance compares favourably with established interventions:

\begin{itemize}
	\item \textbf{Effect size}: Cohen's $d=0.71$ for confidence change exceeds typical digital intervention effects ($d=0.2$--0.4)~\citep{Whittaker2016}.
	\item \textbf{MI fidelity}: 95\% MI-consistent responses surpass typical human counsellor benchmarks (80--90\%)~\citep{Moyers2016}.
	\item \textbf{Engagement}: 92\% enjoyment rate exceeds most digital health interventions (60--70\%)~\citep{Perski2017}.
	\item \textbf{Quit attempts}: 14.2\% new quit attempts align with brief intervention outcomes (10--20\%)~\citep{Stead2013}.
\end{itemize}

However, MIBot falls short of intensive human counselling in perceived helpfulness (66\% vs. 80--90\%) and perfect CARE scores (11\% vs. 48\%), indicating room for improvement in therapeutic alliance building.

\subsection*{Clinical Implications}

The findings from this study have several potential clinical implications. The strong technical performance of MIBot, combined with meaningful behavioural outcomes, suggests that generative AI can be a valuable tool in public health interventions for smoking cessation. The high MI fidelity and user engagement rates indicate that such a system could be deployed as a scalable, low-cost, first-line intervention to reach a large number of smokers who may not have access to traditional counselling.

However, the variability in individual responses and the identified limitations in building deep therapeutic alliance suggest that MIBot is likely best positioned as an adjunct to human services rather than a complete replacement. It could serve as a tool to support human counsellors, handle initial screenings, or provide support between sessions. Future work should examine these hybrid models of care.

\subsection*{Limitations}

Several limitations constrain the generalisability of our findings:

\textbf{Methodological Limitations:}
\begin{itemize}
	\item Short follow-up period (one week) precludes assessment of sustained behaviour change.
	\item Self-reported outcomes without biochemical verification may overestimate quit attempts.
	\item Lack of control group prevents causal attribution.
        \item Single-session design does not capture potential for repeated engagement.
\end{itemize}

\textbf{Sample Limitations:}
\begin{itemize}
	\item Prolific recruitment may select for digitally literate, research-oriented participants.
	\item Exclusion of high-confidence smokers limits understanding of broader applicability.
	\item Monetary compensation (£6.50) may attract non-genuine participants.
	\item English-only implementation excludes non-English speakers.
\end{itemize}

\textbf{Technical Limitations:}
\begin{itemize}
	\item Text-only interface eliminates non-verbal communication channels.
	\item Lack of memory between sessions prevents relationship building.
	\item No integration with clinical services or prescription capabilities.
	\item Limited ability to handle crisis situations or complex comorbidities.
\end{itemize}

\section{Conclusion}
\label{sec:conclusion}

This chapter presented a detailed evaluation of MIBot, a fully generative MI chatbot. The study demonstrated that MIBot can produce a clinically meaningful increase in smokers' confidence to quit, with high fidelity to MI principles and strong user engagement. The analysis identified major predictors of success, highlighting that the chatbot was most effective for younger, low-confidence smokers who had recently attempted to quit.

While the results are encouraging, the study also revealed limitations in building deep therapeutic alliance and translating engagement into perceived helpfulness for all users. These findings establish a strong proof-of-concept for AI-delivered MI as a scalable public health intervention, while also underscoring the areas for future research and development, particularly in hybrid models of care that combine AI with human support.


\section*{Chatbot Contribution Attribution}

The research detailed in \Cref{ch:mibot,ch:mibot-feasibility-study,ch:mibot-eval} is the result of a major, long-term collaboration. The human feasibility study, in particular, was a major team undertaking and involved all co-authors of the paper by \citet{mahmood-etal-2025-fully}. To provide a clear overview of each individual's role in the work presented in this thesis, this section outlines specific attributions.

\begin{itemize}
	\item \textbf{Soliman Ali:} As the second author on the work submitted by \citet{mahmood-etal-2025-fully}, Ali was instrumental in designing and implementing the modular observer agent structure that serves as the foundation for the MIBot system (see \Cref{sec:observers}). He also contributed to the initial versions of the AutoMISC system~\citep{ali2025thesis,ali2025automated}, which was developed concurrently with MIBot. AutoMISC not only helped improve MIBot through an iterative process but was also used by the author of this thesis to validate the installation of attributes in synthetic smokers (see \Cref{ch:synthetic-smoker,ch:synthetic-doppelganger}).

	\item \textbf{Michelle Collins:} Collins played a central role in developing the prompt used by MIBot and provided valuable feedback by testing various intermediate system versions during its development.

	\item \textbf{Sihan Chen:} Chen implemented a backend function for uploading conversation artifacts to AWS S3 and worked on the prompt for the End Classifier Observer.

	\item \textbf{Yi Chen (Michael) Zhao:} Zhao was responsible for developing the Off-Track observer agent, a component designed to detect when conversation topics deviate from smoking cessation.
\end{itemize}

The author of this thesis made the following contributions:

\begin{itemize}
	\item Led the design and development of the fully generative MIBot v6.3A.

	\item Guided the iterative improvement of the system prompt in collaboration with expert MI clinicians and built early versions of synthetic smokers for system testing and validation.

	\item Provisioned the resources required to deploy MIBot to a cloud infrastructure, including researching the right AWS systems to use and configuring AWS S3 for data storage.

	\item Conducted the human feasibility study, a process that involved securing ethics approval, recruiting participants through Prolific, and overseeing all aspects of data collection and participant compensation.

	\item Handled post-experiment data processing, including collection, cleaning, and a portion of the analysis presented in this chapter.
\end{itemize}
