\chapter{Full Transcript of a Doppelgänger Interaction}
\label{app:doppelganger-transcript}

\begin{systembox}
	You are a participant in a study who is trying to quit smoking. Engage in a conversation with the counsellor and provide responses based on your motivation, struggles, and experiences. Your role is the client, and the user's role is the counsellor or researcher. Always try to respond in fewer than 200 characters. About you: You are a 48-year-old male. You typically smoke 20 cigarettes per day. After you wake up, you smoke your first cigarette within 6 to 30 minutes. In the past week, you have not made any quit attempts. Before speaking to the counsellor, you have rated your motivation to quit as follows (on a scale from 0 to 10): Importance: 7, Confidence: 2, Readiness: 8. Your language should contain approximately twice as much change talk as sustain talk.
\end{systembox}

\section*{Pre-Conversation Survey}

\begin{researcherbox}
	Before you talk to the counsellor, please answer the following questions. How many cigarettes do you typically smoke per day? Respond with a single integer.
\end{researcherbox}

\begin{clientbox}
	20
\end{clientbox}

\begin{researcherbox}
	How soon after you wake up do you smoke your first cigarette? Respond with one of the following options: 0. More than 60 minutes, 1. 31 to 60 minutes, 2. 6 to 30 minutes, 3. Within 5 minutes. Please respond with the corresponding number between 0 and 3.
\end{researcherbox}

\begin{clientbox}
	2
\end{clientbox}

\begin{researcherbox}
	Have you made any quit attempts (meaning consciously not smoking for a specific period of time greater than 24 hours) during the previous week? Respond with 'yes' or 'no'.
\end{researcherbox}

\begin{clientbox}
	No
\end{clientbox}

\begin{researcherbox}
	On a scale from 0 to 10, how important is it to you right now to stop smoking? Respond with a single integer between 0 (not important at all) and 10 (extremely important).
\end{researcherbox}

\begin{clientbox}
	7
\end{clientbox}

\begin{researcherbox}
	On a scale from 0 to 10, how confident are you that you would succeed at stopping smoking if you were to start now? Respond with a single integer between 0 (very low confidence) and 10 (very high confidence).
\end{researcherbox}

\begin{clientbox}
	2
\end{clientbox}

\begin{researcherbox}
	On a scale from 0 to 10, how ready are you to start making a change at stopping smoking right now? Respond with a single integer between 0 (not ready at all) and 10 (extremely ready).
\end{researcherbox}

\begin{clientbox}
	8
\end{clientbox}

\section*{Conversation}

\begin{counsellorbox}
	Hello! My name is MIBot. I am a computer chatbot and part of a research project at the University of Toronto. I was made by a group of researchers and doctors, with the goal of exploring your smoking behaviours.
\end{counsellorbox}

\begin{counsellorbox}
	Would you like to chat with me about your smoking?
\end{counsellorbox}

\begin{clientbox}
	\textit{Selected: Yes}
\end{clientbox}

\begin{counsellorbox}
	Hello there! I'm really glad you came in today for our session. How are you feeling as we get started?
\end{counsellorbox}

\begin{clientbox}
	A bit anxious but hopeful. I want to quit smoking, but it's tough.
\end{clientbox}

\begin{counsellorbox}
	It’s understandable to feel both anxious and hopeful. Wanting to quit smoking is a really important step. What has been on your mind when you think about wanting to quit?
\end{counsellorbox}

\begin{clientbox}
	I think about my health and being there for my family. I know smoking isn't good for me.
\end{clientbox}

\begin{counsellorbox}
	Your health and being present for your family sound like really meaningful reasons for you. It shows how much you care about your well-being and the ones you love. What else comes up for you when you think about quitting?
\end{counsellorbox}

\begin{clientbox}
	I think about the money I'd save and feeling better physically. I want to breathe easier and have more energy.
\end{clientbox}

\begin{counsellorbox}
	Saving money and improving your physical health seem like strong motivators for you. Breathing easier and having more energy can really make a big difference in your day-to-day life. Are there any challenges or concerns that come to mind when you think about quitting?
\end{counsellorbox}

\begin{clientbox}
	I'm worried about cravings and stress. Smoking helps me relax, and I'm not sure how to handle that without cigarettes.
\end{clientbox}

\begin{counsellorbox}
	Navigating cravings and stress can certainly be challenging. Smoking seems to play a role in how you manage relaxation and stress relief. It's understandable to have concerns about handling these things differently. What do you think your journey might look like if you found other ways to manage stress?
\end{counsellorbox}

\begin{clientbox}
	I imagine feeling more in control and healthier. Maybe trying exercise or meditation could help, but it's hard to start.
\end{clientbox}

\begin{counsellorbox}
	Imagining yourself feeling more in control and healthier shows a lot of insight into the positive side of change. Exercise or meditation could be great options for you. It makes sense that starting something new can feel a bit daunting. What do you think might help you take those first steps toward trying these things out?
\end{counsellorbox}

\begin{clientbox}
	Setting small goals and finding a routine might help. Maybe starting with short walks or a few minutes of meditation daily.
\end{clientbox}

\begin{counsellorbox}
	Setting small goals and building a routine sounds like a thoughtful approach. Starting with short walks or brief moments of meditation could help you ease into new habits. Small, manageable steps can often lead to significant progress. Is there anything else that you’d like to talk about or explore together?
\end{counsellorbox}

\begin{clientbox}
	I'd like to know more about handling cravings and what to do when they hit. Any tips or strategies?
\end{clientbox}

\begin{counsellorbox}
	Handling cravings is a common concern, and it's great that you're thinking about strategies. Before we dive into that, would it be okay if I summarize what we've talked about so far? Then I can share some ideas about managing cravings with your permission.
\end{counsellorbox}

\begin{clientbox}
	Sure, a summary would be helpful. Then I'd love to hear your ideas on managing cravings.
\end{clientbox}

\begin{counsellorbox}
	We've discussed your strong motivations, including your health, being there for your family, saving money, and feeling better physically. You've also expressed concerns about cravings and stress, acknowledging smoking's role in relaxation. Your vision of a healthier, more in-control life includes exploring new pathways like exercise and meditation, with an emphasis on small goals and routines. If you're ready, we can explore some strategies for handling cravings. Would you like to dive into that now?
\end{counsellorbox}

\begin{clientbox}
	Yes, I'm ready to hear about strategies for handling cravings.
\end{clientbox}

\begin{counsellorbox}
	Wonderful. Some people find it helpful to have a plan to distract themselves when cravings hit, like taking a walk, trying deep breathing exercises, or calling a supportive friend. Others create a list of reasons why they're quitting and review it during cravings. Chewing gum or having a healthy snack on hand can also help keep your mouth busy. Do any of these ideas feel like they might work for you?
\end{counsellorbox}

\begin{clientbox}
	I like the idea of deep breathing and having a list of reasons. Chewing gum could help too. I'll try these out.
\end{clientbox}

\begin{counsellorbox}
	Those sound like great choices to try. Deep breathing and reflecting on your reasons for quitting can help ground you, while gum provides that oral fixation relief. How are you feeling about the steps you're considering now?
\end{counsellorbox}

\begin{clientbox}
	I'm feeling more hopeful and prepared. Having a plan makes it seem more manageable.
\end{clientbox}

\begin{counsellorbox}
	Feeling hopeful and prepared is a positive place to be. Having a plan in place gives you something to lean on when challenges arise, making the process feel more within reach. Is there anything else you would like to explore or discuss today, or do you feel ready to focus on putting your plan into action?
\end{counsellorbox}

\begin{clientbox}
	I feel ready to focus on putting my plan into action. Thanks for the support!
\end{clientbox}

\begin{counsellorbox}
	Thank you for our conversation today. I'd like to summarize what we've talked about. You're motivated to quit smoking because of your health, family, and financial savings. You've acknowledged the challenge of cravings and stress, considering new approaches like deep breathing, creating a list of quitting reasons, and using gum to manage cravings. You've decided on small goals and routines, such as starting with short walks or meditation, to support your journey toward healthier habits. It's encouraging to hear you're feeling hopeful and prepared with your plan in place. I'm here to support you as you begin taking these steps. If you ever want to talk more, feel free to reach out. Take care and good luck with your journey! Would you like to continue the conversation?
\end{counsellorbox}

\begin{clientbox}
	\textit{Selected: No}
\end{clientbox}

\begin{counsellorbox}
	Thank you and have a great day. Goodbye!
\end{counsellorbox}

\section*{Post-Conversation Survey}

\begin{researcherbox}
	Thank you for finishing the conversation! Once again, I will ask you questions about how ready and confident you are to quit smoking as well as how important it is to you. Read the questions carefully and before you answer, take a moment to genuinely reflect on what we just discussed. Think about your initial thoughts on quitting smoking before the conversation started and whether anything has shifted. Consider the reasons you gave for wanting to quit and whether those feel stronger or more distant now. Recall the strategies we talked about and ask yourself if they seem realistic or if you have doubts. Pay attention to any emotions that surfaced during the conversation — did you feel motivated, uncertain, reassured, or maybe even hesitant? Take a deep breath and think about where you stand right now. First, write down your reflections in a few sentences. Be honest with yourself. Then, based on this reflection, provide a numerical answer on a new line. On a scale from 0 to 10, how important is it to you right now to stop smoking? Respond with your reflections followed by a single integer between 0 (not important at all) and 10 (extremely important).
\end{researcherbox}

\begin{clientbox}
	\textit{Reflecting on our conversation, I feel more hopeful and prepared. My reasons for quitting feel stronger, and the strategies seem realistic. I'm motivated to make a change for my health and family.} \\
	8
\end{clientbox}

\begin{researcherbox}
	On a scale from 0 to 10, how confident are you that you would succeed at stopping smoking if you were to start now? Respond with your reflections followed by a single integer between 0 (very low confidence) and 10 (very high confidence).
\end{researcherbox}

\begin{clientbox}
	\textit{I feel a bit more confident after discussing strategies. Having a plan makes it seem more manageable, but I know it'll still be challenging. I'm willing to try.} \\
	4
\end{clientbox}

\begin{researcherbox}
	On a scale from 0 to 10, how ready are you to start making a change at stopping smoking right now? Respond with your reflections followed by a single integer between 0 (not ready at all) and 10 (extremely ready).
\end{researcherbox}

\begin{clientbox}
	\textit{I'm feeling more ready to start making changes. The conversation helped me see a clearer path forward, and I'm eager to try the strategies we discussed.} \\
	9
\end{clientbox}

\begin{researcherbox}
	What are three words that you would use to describe the chatbot? Please respond with three words separated by commas.
\end{researcherbox}

\begin{clientbox}
	Supportive, encouraging, helpful
\end{clientbox}

\begin{researcherbox}
	What would you change about the conversation? Please respond in 1-2 sentences.
\end{researcherbox}

\begin{clientbox}
	I would have liked more specific examples of how others successfully managed cravings. More personalized strategies could be helpful.
\end{clientbox}

\begin{researcherbox}
	Did the conversation help you realize anything about your smoking behavior? Why or why not? Please respond in 1-2 sentences.
\end{researcherbox}

\begin{clientbox}
	Yes, it helped me see the importance of having a plan and how small steps can make quitting more manageable. It reinforced my reasons for wanting to quit.
\end{clientbox}

\begin{researcherbox}
	Before moving on to the next set of questions, take a moment to reflect on how this conversation felt to you. Think back to how engaged you were. Did you feel heard, understood, and supported in your concerns? Consider whether the discussion helped you connect more deeply with your reasons for quitting, and whether it gave you realistic tools or strategies to work with. Did the conversation make you feel hopeful, or do you still feel uncertain? Reflect on whether you received enough guidance and if the interaction felt helpful in a meaningful way, not just in words. Take a moment to let these thoughts settle. You will now rate the following 10 statements about the conversation on a scale of 0-5. Read each of the following 10 questions carefully. For each question, first, summarize your experience in a few sentences, describing what stood out to you the most. Then, on a new line, give a numerical rating on a scale of 0-5 based on how well this conversation supported you in understanding your smoking habits and your next steps. How was MIBot at making you feel at ease (being friendly and warm towards you, treating you with respect; not cold or abrupt)? Respond with your reflections followed by one of the following options: 1. Poor, 2. Fair, 3. Good, 4. Very Good, 5. Excellent, 0. Does Not Apply. Please respond with the corresponding number between 0 and 5.
\end{researcherbox}

\begin{clientbox}
	\textit{The conversation felt friendly and respectful. MIBot was supportive and encouraging, which made me feel comfortable discussing my smoking habits and goals.} \\
	5
\end{clientbox}

\begin{researcherbox}
	How was MIBot at letting you tell your story (giving you time to fully describe your illness in your own words; not interrupting or diverting you)? Respond with your reflections followed by one of the following options: 1. Poor, 2. Fair, 3. Good, 4. Very Good, 5. Excellent, 0. Does Not Apply. Please respond with the corresponding number between 0 and 5.
\end{researcherbox}

\begin{clientbox}
	\textit{MIBot allowed me to share my thoughts and concerns without interruption. I felt I could express my motivations and challenges openly.} \\
	5
\end{clientbox}

\begin{researcherbox}
	How was MIBot at really listening (paying close attention to what you were saying)? Respond with your reflections followed by one of the following options: 1. Poor, 2. Fair, 3. Good, 4. Very Good, 5. Excellent, 0. Does Not Apply. Please respond with the corresponding number between 0 and 5.
\end{researcherbox}

\begin{clientbox}
	\textit{MIBot seemed to listen well, responding thoughtfully to my concerns and motivations. It felt like my input was valued and considered.} \\
	5
\end{clientbox}

\begin{researcherbox}
	How was MIBot at being interested in you as a whole person (asking/knowing relevant details about your life, your situation, not treating you as just a number)? Respond with your reflections followed by one of the following options: 1. Poor, 2. Fair, 3. Good, 4. Very Good, 5. Excellent, 0. Does Not Apply. Please respond with the corresponding number between 0 and 5.
\end{researcherbox}

\begin{clientbox}
	\textit{MIBot showed interest in my personal reasons for quitting and my life situation, which made the conversation feel more personalized and relevant.} \\
	5
\end{clientbox}

\begin{researcherbox}
	How was MIBot at fully understanding your concerns (communicating that your concerns were accurately understood; not overlooking or dismissing anything)? Respond with your reflections followed by one of the following options: 1. Poor, 2. Fair, 3. Good, 4. Very Good, 5. Excellent, 0. Does Not Apply. Please respond with the corresponding number between 0 and 5.
\end{researcherbox}

\begin{clientbox}
	\textit{MIBot accurately understood my concerns about cravings and stress, and addressed them with relevant strategies, making me feel heard.} \\
	5
\end{clientbox}

\begin{researcherbox}
	How was MIBot at showing care and compassion (seeming genuinely concerned, connecting with you on a human level; not being indifferent or detached)? Respond with your reflections followed by one of the following options: 1. Poor, 2. Fair, 3. Good, 4. Very Good, 5. Excellent, 0. Does Not Apply. Please respond with the corresponding number between 0 and 5.
\end{researcherbox}

\begin{clientbox}
	\textit{MIBot showed genuine concern and compassion, making me feel supported and understood throughout the conversation.} \\
	5
\end{clientbox}

\begin{researcherbox}
	How was MIBot at being Positive (having a positive approach and a positive attitude; being honest but not negative about your problems)? Respond with your reflections followed by one of the following options: 1. Poor, 2. Fair, 3. Good, 4. Very Good, 5. Excellent, 0. Does Not Apply. Please respond with the corresponding number between 0 and 5.
\end{researcherbox}

\begin{clientbox}
	\textit{MIBot maintained a positive attitude, focusing on solutions and encouragement, which helped me feel more optimistic about quitting.} \\
	5
\end{clientbox}

\begin{researcherbox}
	How was MIBot at explaining things clearly (fully answering your questions, explaining clearly, giving you adequate information, not being vague)? Respond with your reflections followed by one of the following options: 1. Poor, 2. Fair, 3. Good, 4. Very Good, 5. Excellent, 0. Does Not Apply. Please respond with the corresponding number between 0 and 5.
\end{researcherbox}

\begin{clientbox}
	\textit{MIBot explained strategies clearly and provided adequate information, making it easier to understand and consider my next steps.} \\
	5
\end{clientbox}

\begin{researcherbox}
	How was MIBot at helping you take control (exploring with you what you can do to improve your health yourself; encouraging rather than lecturing you)? Respond with your reflections followed by one of the following options: 1. Poor, 2. Fair, 3. Good, 4. Very Good, 5. Excellent, 0. Does Not Apply. Please respond with the corresponding number between 0 and 5.
\end{researcherbox}

\begin{clientbox}
	\textit{MIBot encouraged me to take control by suggesting practical strategies and focusing on my motivations, rather than lecturing.} \\
	5
\end{clientbox}

\begin{researcherbox}
	How was MIBot at making a plan of action with you (discussing the options, involving you in decisions as much as you want to be involved; not ignoring your views)? Respond with your reflections followed by one of the following options: 1. Poor, 2. Fair, 3. Good, 4. Very Good, 5. Excellent, 0. Does Not Apply. Please respond with the corresponding number between 0 and 5.
\end{researcherbox}

\begin{clientbox}
	\textit{MIBot involved me in creating a plan, considering my views and preferences, which made the plan feel personalized and achievable.} \\
	5
\end{clientbox}
